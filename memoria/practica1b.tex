%Encabezado estándar
\documentclass[10pt,a4paper]{article}
\usepackage{pgfplots}
\usepackage[utf8]{inputenc}
\usepackage{amsmath}
\usepackage{amssymb}
\usepackage{amsthm}
\usepackage{hyperref}
\usepackage{graphicx}
\usepackage{subfigure} %paquete para poder añadir subfiguras a una figura
\usepackage{listings}
\usepackage{color}
\usepackage{float}
\usepackage[toc,page]{appendix} %paquete para hacer apéndices
\usepackage{cite} %paquete para que Latex contraiga las referencias [1-4] en lugar de [1][2][3][4]
\usepackage[nonumberlist]{glossaries} %[toc,style=altlistgroup,hyperfirst=false] 
%usar makeglossaries grafo para recompilar el archivo donde están los grafos y que así salga actualizado
\author{Gustavo Rivas Gervilla DNI: 75570417F \\ gustavofox92@correo.ugr.es \\5º Doble Grado en Ing. Informática y Matemáticas \\Grupo de los Viernes de 17:30 a 19:30}
\title{Práctica 1.b: Búsquedas por Trayectorias para el Problema de la Selección de Características \\ SFS \\ Búsqueda Local de Primero el Mejor \\ Enfriamiento Simulado \\ Búsqueda Tabú}
\date{}

%Configuración especial
\setlength{\parindent}{0cm}
\pretolerance=10000
\tolerance=10000
\renewcommand{\contentsname}{\color[rgb]{0.0,0.0,0.21}Índice}

\hypersetup{
  colorlinks=true,%colorear el texto en lugar de poner una caja de color alrededor
  citecolor=orange,%citas bibliográficas, del estilo [8]
  urlcolor=orange,%urls
  linkcolor=[rgb]{0.0,0.0,0.21}}%links internos como los del índice
  
\lstdefinestyle{customPy}{
  belowcaptionskip=1\baselineskip,
  breaklines=true,
  frame=L,
  xleftmargin=\parindent,
  language=Python,
  showstringspaces=true,
  basicstyle=\footnotesize\ttfamily,
  keywordstyle=\bfseries\color{green!40!black},
  commentstyle=\itshape\color{purple!40!black},
  identifierstyle=\color{blue},
  stringstyle=\color{orange},
}

\begin{document}
\lstset{language=Python, style=customPy}
\maketitle

\newpage

\tableofcontents

\newpage

\section{\color[rgb]{0.0,0.0,0.21}Descripción/formulación del problema abordado}
Lo que intentamos hacer con nuestro algoritmos es encontrar un conjunto de características de unos datos que nos permitan realizar una clasificación suficientemente buena de nuevos datos que nos lleguen con las mismas características.\\

Hay un problema muy habitual en la vida real y es la de clasificar una serie de elementos en distintas categorías en función de información sobre ellos, esta tarea puede ser realizada por personas o, lo que es más eficiente, por un ordenador. Para realizar tal clasificación es habitual que se recojan multitud de datos sobre los distintos elementos que se quieren clasificar, de modo que en base a esta información podamos decidir si el elemento es de una categoría o de otra. Pensemos por ejemplo en clasificar fruta en base a si se desecha o no. Podemos pensar en recoger datos sobre el tamaño de esa fruta, su color, su textura o su dureza y en base a estas mediciones una máquina debería clasificar la fruta en buena o mala.\\

El problema está en que normalmente no se conoce tan bien el campo de estudio como para saber a ciencia cierta qué datos recoger, qué datos serán más relevantes a la hora de clasificar elementos de una determinada población. Entonces lo que se hace es recoger gran cantidad de información sobre los elementos para al menos intentar que no haya carencias en la información, esto por supuesto conlleva tanto el coste de adquirir esa información (no sabemos cómo de caro es realizar una determinada medición) y el coste computacional de procesar toda esa información. Entonces lo que nos gustaría es averiguar qué información, de entre toda la que hemos obtenido, es la verdaderamente relevante para la clasificación que queremos realizar.\\

Entonces partiendo de un conjunto de datos de aprendizaje, valores de características de distintos elementos, queremos ver con qué subconjunto de características podemos hacer una buena clasificación de esos elementos, así si tenemos que cada dato viene dado por una lista de n características $[f_1, f_2, ..., f_n]$ queremos obtener un subconjunto de esas características, de modo que teniendo sólo la información $[f_{s1}, ..., f_{sm}]$, se haga una buena clasificación del conjunto de datos de aprendizaje, del que conocemos por supuesto la clasificación perfecta de dichos datos. Y esperamos que con esa misma información se clasifiquen lo mejor posible nuevos elementos de fuera de la muestra de aprendizaje.\\

\newpage

\section{\color[rgb]{0.0,0.0,0.21}Descripción de la aplicación de los algoritmos empleados al problema}

Dado que estamos ante un problema de selección las soluciones se representarán como vectores binarios de booleanos de tamaño el número de características a elegir, indicando si una característica se considera o no, así tendremos claramente un espacio de búsqueda de $2^n$, siendo el n el número de características a elegir que también lo podemos ver como el tamaño del problema abordado.\\

Entonces para evaluar como de buena es una determinada solución hacemos lo siguiente:\\

\begin{lstlisting}
tomamos de cada dato de entrenamiento las caracteristicas seleccionada por la solucion.

for cada dato de entrenamiento:
	clasificador_dato = construir clasificador 3NN con el resto de datos
	ver si clasificador_dato clasifica correctamente a ese dato
	
return media del numero de aciertos
\end{lstlisting}

Para generar las soluciones vecinas a una dada usamos el operador flip que funciona del siguiente modo:\\

\begin{lstlisting}
def flip(sol, idx):
	cambiar el valor de la pos. idx de la sol por su negado
\end{lstlisting}

Cuando queramos ver cómo de buena es la solución final que ha obtenido nuestro algoritmo de búsqueda lo que hacemos es:\\

\begin{lstlisting}
claficador = construir clasificador con los datos de entrenamiento solo con las caracterisiticas seleccionadas por nuestra solucion
ver el porcentaje de acierto de este clasificador etiquetando los datos de test
\end{lstlisting}

A cada algoritmo le daremos solamente los datos de entrenamiento separados en características y etiquetas pera que con la estragia de búsqueda que implemente nos devuelva la mejor solución posible para él, luego la evaluación de la solución final la haremos fuera del algoritmo con los datos de test (en la sección 8 ya diremos cómo hemos generado las particiones TODO).
\newpage
\section{\color[rgb]{0.0,0.0,0.21}Descripción en pseudocódigo de los algoritmos}
\subsection{\color[rgb]{0.0,0.0,0.51}Búsqueda Local (BL.py)}

\begin{lstlisting}
s = solucion inicial aleatoria

while True:
	for cada vecino de s explorados en orden aleatorio:
	
		if el vecino es mejor que s:
			s = el vecino
		
		if se ha generado el numero max total de sols:
			return s
			
		if se ha encontrado vecino mejor a la sol actual:
			break
			
	if no hemos encontrado ninguna solucion mejor en todo el vecindario:
		return s
\end{lstlisting}

\newpage
\subsection{\color[rgb]{0.0,0.0,0.51}Enfriamiento simulado (ES.py)}

El esquema de enfriamiento que seguimos es el de Cauchy modificado para el que calculamos la nueva temperatura a partir de la anterior del siguiente modo:\\

\begin{lstlisting}
nueva_T = T_actual/(1+beta*T_actual) #beta es una param. del problema
\end{lstlisting}

y la temperatura inicial se calcula en base al "costo" de la solución inicial, que en este caso es su poder de clasificación para las muestras de entrenamiento de las que partamos, y dos parámtros $\mu$ y $\phi$ que fijamos a 0.3 con lo que el cálculo de la temperatura inicial es tan sencillo como el siguiente:\\

\begin{lstlisting}
T0 = -mu*coste_sol_inicial/log(fi)
\end{lstlisting}

hay algo que me sorprende de este hecho y es que dado que el logarirmo neperiano de 0.3 es negativo, en nuestro cálculo, cuanto mejor sea la solución inicial de la que partamos, más elevada será dicha temperatura, es decir, más tardaremos en llegar a la temperatura final.\\

Pasamos ya a describir el esquema de búsqueda del algoritmo para el enfriamiento simulado:\\

\begin{lstlisting}
generamos una sol. inicial aleatoria y calculamos los params. de el problema

while se acepte algun vecino en el vecindario explorado de la sol actual and no se hayan generado 15000 vecinos en total:

	n_exitos = 0 #vecinos generados y aceptados hasta el momento
	for hasta el max_vecinos permitidos:
		generamos un vecino aleatorio, cambiando una caracteristica de la sol actual al azar
		n_vecinos_generados++
		
		delta = diferencia entre coste sol. actual y el vecino
		
		if (el vecino es mejor que la sol actual or U(0,1) <= exp(-delta/temperatura_actual) #U(0,1) valor aleatorio uniforme en el [0,1]) and delta != 0:
			sol actual = vecino
			n_exitos++
		
			if la nueva sol es mejor que la mejor sol encontrada hasta el momento:
				mejor_sol = sol_actual
			
		if n_exitos == max_exitos or hemos generado el numero total de vecinos posibles:
			break
			
	actualizar temperatura	
	
return la mejor solucion encontrada
\end{lstlisting}

Hemos añadido la condición de que delta no sea cero ya que de otro modo la exponencial que usamos para aceptar soluciones peores valdría uno con lo que estaríamos tomando siempre aquellos vecinos con el mismo coste que la solución actual, esto hacía que el algoritmo generase siempre los 15000 vecinos que tiene como tope, tardando media hora para una partición de wdbc.\\

\newpage
\subsection{\color[rgb]{0.0,0.0,0.51}Búsqueda Tabú (BT.py)}

Aquí tenemos el pseudocódigo para la búsqueda tabú básica, en el también se ve cómo manejamos la lista tabú como una lista cíclica que consultamos para ver si un vecino ha sido generado mediante un movimiento tabú:\\

\begin{lstlisting}
while no se hayan generado el maximo de vecinos permitido:
	elegimos n vecinos aleatorios a revisar, siendo n el maximo permitido por iteracion

	for cada vecino elegido:

		if el movimiento que lo genera esta en la lista tabu(LT):
			if el vecino es mejor que la mejor sol encontrada and mejor que el mejor vecino explorado en esta iteracion:
				mejor vecino = nuevo vecino
		else:
			if mejor que el mejor vecino explorado en esta iteracion:
				mejor vecino = nuevo vecino

		if hemos generado el maximo de vecinos totales permitidos:
			break

	sol_actual = mejor vecino encontrado en la iteracion

	if sol_actual mejor que la mejor encontrada hasta el momento:
		mejor_sol_encontrada = sol_actual

	LT[posicion_correspondiente] = movimiento que genero el mejor vecino de esta iteracion
	posicion_correspondiente = a la siguiente ciclicamente

return mejor_sol_encontrada
\end{lstlisting}

\newpage
\subsection{\color[rgb]{0.0,0.0,0.51}Búsqueda Tabú Extendida (BText.py)}

\newpage
\section{\color[rgb]{0.0,0.0,0.21}Breve descripción del algoritmo de comparación}

Para el algoritmo de comparación, SFS, lo que hacemos es lo que vamos a reflejar en el siguiente pseudocódigo, la principal diferencia es que no tenemos la función usual flip sino que hemos hecho una función especial para poder realizar una función vectorizada en Python (aunque no produce ninguna mejora en tiempo a la versión que tendríamos si simplemente tuviésemos un for que recorriese las características que quedan por añadir hasta encontra la de mayor ganancia), esta función crea una nueva solución poniendo a True una componente de la solución que le pasamos y nos da su porcentaje de clasificación, como hacemos en el resto de algoritmos. Dicho esto pasamos al pseudocódigo:

\begin{lstlisting}
sol = un array binario con todo False #conjunto vacio de caracteristicas

while tengamos ganancia and queden caracteristicas por aniadir:
	calcular el score de cada caracteristica por aniadir al agregarla al conjunto actual
	tomar la caracteristica que de mejor score en el calculo anterior
	
	if el score aniadiendola es mejor que el de el conjunto mejor al que habia:
		se agrega dicha caracteristica al conjunto
		quitamos esa caracteristica de el conjunto de caracteristicas por aniadir
	else:
		no tenemos ganancia y acabamos el bucle
		
return el conjunto de caracteristicas al que hemos llegado
\end{lstlisting}
\newpage
\section{\color[rgb]{0.0,0.0,0.21}Procedimiento considerado para desarrollar la práctica}

\end{document}